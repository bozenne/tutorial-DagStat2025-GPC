\documentclass[12pt]{article}
\usepackage{amssymb}
%\usepackage{amsmath}
\usepackage{graphicx}
\usepackage{subfigure}
%\usepackage{epsfig}
\usepackage{makeidx}
%\usepackage{showidx}
\usepackage{multicol}
\usepackage{natbib}
\bibliographystyle{plainnat}

\usepackage{amsmath,bm}
\usepackage{amsthm}
\usepackage{amsfonts}
\usepackage{latexsym}
\usepackage{multirow}
\usepackage{nicefrac}
\usepackage{geometry}
\geometry{top=1cm,left=4cm,right=4cm}
\usepackage{mathrsfs,bbm,enumerate}
\theoremstyle{definition}
\newtheorem{defn}{Definition}
\newtheorem{expl}{Example}


\usepackage{dsfont} %% font
\usepackage{url} %% refer to url
\usepackage{listings}  %% code output
\usepackage{fancyvrb}  %% code output
\usepackage[dvipsnames]{xcolor}

\definecolor{colour_pink}{rgb}{1,0,1}
\newcommand{\darkgreen}{green!70!black}
\newcommand{\darkblue}{blue!80!black}
\newcommand{\Ysurv}{\textcolor{\darkblue}{S}}
\newcommand{\yobs}{\textcolor{\darkblue}{\tilde{y}_j}}
\newcommand{\Ycens}{\textcolor{\darkblue}{\omega_j}}

\newcommand{\Xsurv}{\textcolor{orange}{S}}
\newcommand{\xobs}{\textcolor{orange}{\tilde{y}_i}}
\newcommand{\Xcens}{\textcolor{orange}{\omega_i}}



% R code
\definecolor{colour_gray}{rgb}{0.7,0.7,0.7}
\definecolor{colour_Rcomment}{rgb}{0,0.6,0}
\definecolor{colour_Rline}{rgb}{0.5,0.5,0.5}
\definecolor{colour_Rbold}{rgb}{0,0,0}
\definecolor{colour_Rcode}{rgb}{0.58,0,0.82}
\definecolor{colour_Rbackinput}{rgb}{0.95,0.95,0.92}
\lstdefinestyle{Rstyle}{
    backgroundcolor=\color{colour_Rbackinput},   
    commentstyle=\color{colour_Rcomment},
    keywordstyle=\color{colour_Rbold},
    numberstyle=\tiny\color{colour_Rline},
    stringstyle=\color{colour_Rbackinput},
    basicstyle=\ttfamily\footnotesize,
    breakatwhitespace=false,         
    breaklines=true,                 
    captionpos=b,                    
    keepspaces=true,                 
    numbers=left,                    
    numbersep=5pt,                  
    showspaces=false,                
    showstringspaces=false,
    showtabs=false,                  
    tabsize=2
}
\lstset{style=Rstyle}
\DefineVerbatimEnvironment{Rcode}{Verbatim}{frame=single,label=R code,fontsize=\small,formatcom = {\color[rgb]{0,0,0.5}}}
\DefineVerbatimEnvironment{Routput}{Verbatim}{frame=single,label=R output,fontsize=\small,formatcom = {\color[rgb]{0.5,0,0}}}
% https://texdoc.org/serve/fancyvrb/0

\usepackage[skins,theorems]{tcolorbox}
\tcbset{highlight math style={enhanced,
		colframe=red,colback=white,arc=0pt,boxrule=1pt}}

\begin{document}

\begin{Rcode}
> install.packages("BuyseTest", quiet = TRUE)
> library(BuyseTest)
\end{Rcode}
\begin{Routput}
Loading required package: Rcpp
BuyseTest version 3.1.0
\end{Routput}

\begin{Rcode}
> set.seed(10)
> data <- simBuyseTest(100, n.strata = 2)
> head(data)
\end{Rcode}
\begin{Routput}
      id treatment  eventtime status toxicity      score strata
   <num>    <fctr>      <num>  <num>   <fctr>      <num> <fctr>
1:     1         C 0.17392093      1      yes -2.1250686      a
2:     2         C 0.16255166      0      yes  0.5211787      a
3:     3         C 0.08302502      1      yes -0.0464229      b
4:     4         C 0.22204972      0       no -1.1494717      b
5:     5         C 0.11669726      1       no  0.6293383      a
6:     6         C 0.11885540      1      yes -0.7264715      a
\end{Routput}


\begin{Rcode}
> e.BT <- BuyseTest(treatment ~ tte(eventtime, status = status), 
                    data = data)
\end{Rcode}
\begin{Routput}
         Generalized Pairwise Comparisons

Settings 
- 2 groups  : Control = C and Treatment = T
- 1 endpoint: 
priority endpoint   type           operator             event       
1        eventtime  time to event  higher is favorable  status (0 1)
- right-censored pairs: probabilistic score based on the survival curves 

Point estimation and calculation of the iid decomposition

Estimation of the estimator's distribution 
- method: moments of the U-statistic

Gather the results in a S4BuyseTest object 
\end{Routput}

\clearpage

\begin{Rcode}
> summary(e.BT)
\end{Rcode}
\begin{Routput}
	Generalized pairwise comparisons with 1 endpoint

- statistic       : net treatment benefit  (delta: endpoint specific, Delta: global) 
- null hypothesis : Delta == 0 
- confidence level: 0.95 
- inference       : H-projection of order 1 after atanh transformation 
- treatment groups: T (treatment) vs. C (control) 
- censored pairs  : probabilistic score based on the survival curves
- results
endpoint total(%) favorable(%) unfavorable(%) neutral(%) uninf(%)  Delta CI [2.5% ; 97.5%] p.value 
eventtime     100        57.39          42.61          0        0 0.1479  [-0.0293;0.3161] 0.10151 

\end{Routput}

\begin{Rcode}
> confint(e.BT, statistic = "winRatio")
\end{Rcode}
\begin{Routput}
          estimate        se  lower.ci upper.ci null   p.value
eventtime 1.347081 0.2450411 0.9430953 1.924118    1 0.1014458
\end{Routput}

\begin{Rcode}
> e.BThalf <- BuyseTest(treatment ~ tte(eventtime, status),
                        data = data, add.halfNeutral = TRUE, trace = FALSE)
> model.tables(e.BThalf, statistic = "favorable")
\end{Rcode}
\begin{Routput}
   endpoint total favorable unfavorable neutral uninf     Delta  lower.ci  upper.ci   p.value
1 eventtime   100  57.39388    42.60612       0     0 0.5739388 0.4852354 0.6581263 0.1019135
\end{Routput}

\begin{Rcode}
> coef(e.BThalf, statistic = "winRatio")
\end{Rcode}
\begin{Routput}
[1] 1.347081
\end{Routput}

\clearpage

\begin{Rcode}
> e.MBT <- BuyseTest(treatment ~ tte(eventtime, status, threshold = 1) + bin(toxicity, operator = "<0"),
                     data = data, trace = 0)
> model.tables(e.MBT)
\end{Rcode}
\begin{Routput}
   endpoint threshold total favorable unfavorable neutral uninf   delta  Delta lower.ci upper.ci p.value
1 eventtime     1e+00 100.0      10.2        2.55    87.2     0  0.0768 0.0768 -0.00928    0.162  0.0803
3  toxicity     1e-12  87.2      18.8       24.72    43.7     0 -0.0590 0.0178 -0.13396    0.169  0.8192
\end{Routput}

\begin{Rcode}
plot(e.MBT)
\end{Rcode}

\begin{Rcode}
plot(e.MBT, type = "racetrack")
\end{Rcode}

\begin{Rcode}
> e.BTindiv <- BuyseTest(treatment ~ tte(eventtime, status = status),
                       data = data, keep.pairScore = TRUE)
> getPairScore(e.BTindiv)
\end{Rcode}
\begin{Routput}
       index.C index.T favorable unfavorable neutral        uninf weight
    1:       1     101 0.9192694  0.08073064       0 0.000000e+00      1
    2:       2     101 0.5695583  0.43044167       0 1.110223e-16      1
    3:       3     101 1.0000000  0.00000000       0 0.000000e+00      1
    4:       4     101 0.4969601  0.50303994       0 0.000000e+00      1
    5:       5     101 1.0000000  0.00000000       0 0.000000e+00      1
   ---                                                                  
 9996:      96     200 0.2858328  0.71416716       0 0.000000e+00      1
 9997:      97     200 0.8120919  0.18790807       0 0.000000e+00      1
 9998:      98     200 0.6171644  0.38283561       0 0.000000e+00      1
 9999:      99     200 0.6171644  0.38283561       0 0.000000e+00      1
10000:     100     200 0.4596044  0.54039560       0 0.000000e+00      1
\end{Routput}


\begin{Rcode}
> eBT.perm <- BuyseTest(treatment ~ cont(score), data = data,
                      method.inference = "varexact permutation")
> model.tables(eBT.perm)
\end{Rcode}
\begin{Routput}
  endpoint total favorable unfavorable neutral uninf  Delta   p.value
1    score   100     53.67       46.33       0     0 0.0734 0.3698664
\end{Routput}


\begin{Rcode}
> wilcox.test(score ~ treatment, data = data, correct = FALSE)$p.value
\end{Rcode}
\begin{Routput}
0.3698664
\end{Routput}

\clearpage

\begin{Rcode}
> rbind(confint(e.BTindiv, transformation = TRUE),
        confint(e.BTindiv, transformation = FALSE)) 
\end{Rcode}
       
\begin{Routput}
            estimate         se    lower.ci  upper.ci null    p.value
eventtime  0.1478776 0.08897931 -0.02931684 0.3160612    0 0.10150573
eventtime1 0.1478776 0.08897931 -0.02651861 0.3222739    0 0.09652625
\end{Routput}

\begin{Rcode}
> NTB <- coef(e.BTindiv)
> sigma.NTB <- sqrt(crossprod(getIid(e.BTindiv)))
> sigmaTrans.NTB <- sigma.NTB/(1-NTB^2)
> c(estimate = NTB, se = sigmaTrans.NTB, p.value = 2*(1-pnorm(NTB/sigma.NTB)),
    pTrans.value = 2*(1-pnorm(atanh(NTB)/sigmaTrans.NTB)))
\end{Rcode}

\begin{Routput}
  estimate           se      p.value pTrans.value 
0.14787764   0.09096860   0.09652625   0.10150573 
\end{Routput}


\begin{Rcode}
BuyseTest.options(method.inference = "permutation", n.resampling = 1000,
                  statistic = "winRatio")
\end{Rcode}


\begin{Rcode}
> data("prodige", package = "BuyseTest")
> head(prodige)
\end{Rcode}

\begin{Routput}
    id treatment     OS statusOS    PFS statusPFS toxicity    sex
<num>    <fctr>  <num>    <num>  <num>     <num>    <num> <fctr>
1:     1         C 0.0349        1 0.0349         0        1      F
2:     2         C 2.2790        0 2.2052         1        4      F
3:     3         C 0.2008        1 0.2008         0        1      M
4:     4         C 0.3418        1 0.3418         0        1      F
\end{Routput}

\begin{Rcode}
> e.BR <- BuyseTest(treatment ~ tte(OS, statusOS, threshold = 6)
                + cont(toxicity, operator = "<0", threshold = 2) 
                + tte(OS, statusOS, threshold = 1) 
                + cont(toxicity, operator = "<0"), 
                    data = prodige)
> plot(e.BR)
\end{Rcode}

\begin{Rcode}
> summary(e.BR)
\end{Rcode}

\begin{Rcode}
> M.threshold <- cbind(OS_t6 = c(3:5,3:5),
                       toxicity_t2 = c(2,2,2,3,3,3),
                       OS_t2 = 1,
                       toxicity = 0)
> M.threshold
\end{Rcode}
\begin{Routput}
     OS_t6 toxicity_t2 OS_t2 toxicity
[1,]     3           2     1        0
[2,]     4           2     1        0
[3,]     5           2     1        0
[4,]     3           3     1        0
[5,]     4           3     1        0
[6,]     5           3     1        0
\end{Routput}

\begin{Rcode}
> eBR.Se <- sensitivity(e.BR, band = TRUE,
                      threshold = M.threshold)
> plot(eBR.Se)
\end{Rcode}


\begin{Rcode}
> e.NTB_Gehan <- BuyseTest(treatment ~ tte(OS, statusOS), scoring.rule = "Gehan", 
                         data = prodige, keep.pairScore = TRUE, trace = FALSE)
> getPairScore(e.NTB_Gehan)[1:2,]
\end{Rcode}
\begin{Routput}
   index.C index.T favorable unfavorable neutral uninf weight
1:       1     403         1           0       0     0      1
2:       2     403         0           0       0     1      1
\end{Routput}

\begin{Rcode}
> e.NTB_Peron <- BuyseTest(treatment ~ tte(OS, statusOS), scoring.rule = "Peron", 
                         data = prodige, keep.pairScore = TRUE, trace = FALSE)
> getPairScore(e.NTB_Peron)[1:2,]
\end{Rcode}
\begin{Routput}
   index.C index.T favorable unfavorable neutral        uninf weight
1:       1     403 1.0000000     0.00000       0 0.0000000000      1
2:       2     403 0.5286551     0.47068       0 0.0006648516      1
\end{Routput}


\begin{Rcode}
> e.NTB_Latta <- BuyseTest(treatment ~ tte(OS, statusOS), scoring.rule = "Peron", 
                           data = prodige, trace = FALSE,
                           model.tte = prodlim(Hist(OS, statusOS) ~ 1, data = prodige))
\end{Rcode}



\begin{Rcode}
> e.NTB_restricted <- BuyseTest(treatment ~ tte(OS, statusOS, restriction = 24), 
                                scoring.rule = "Peron", data = prodige)
\end{Rcode}


\clearpage

\begin{Rcode}
> simFCT <- function(n.C, n.T){
   out <- rbind(data.frame(Y=stats::rt(n.C, df = 5), group=0),
                data.frame(Y=stats::rt(n.T, df = 5) + 0.5, group=1)
                )
   return(out)
}
> set.seed(10)
> simFCT(2,2)
\end{Rcode}
\begin{Routput}
            Y group
1  0.02241932     0
2 -1.07273566     0
3  1.26072274     1
4  0.24187644     1
\end{Routput}

\begin{Rcode}
> e.power <- powerBuyseTest(formula = group ~ cont(Y),
                           sim = simFCT, sample.size = c(10,25,50),
                           n.rep = 100, seed = 10, cpus = 1)
> summary(e.power)
\end{Rcode}

\begin{Routput}
         Simulation study with Generalized pairwise comparison
with 100 samples

- net benefit statistic (null hypothesis Delta=0)
endpoint threshold n.T n.C mean.estimate sd.estimate mean.se rejection.rate
       Y     1e-12  10  10        0.2258      0.2651  0.2475           0.12
                    25  25        0.2297      0.1455  0.1585           0.23
                    50  50        0.2331      0.1089  0.1118           0.57

n.T          : number of observations in the treatment group
n.C          : number of observations in the control group
mean.estimate: average estimate over simulations
sd.estimate  : standard deviation of the estimate over simulations
mean.se      : average estimated standard error of the estimate over simulations
rejection    : frequency of the rejection of the null hypothesis over simulations
(standard error: H-projection of order 1| p-value: after transformation) 
\end{Routput}


\begin{Rcode}
> e.n <- powerBuyseTest(formula = group ~ cont(Y),
                        sim = simFCT, power = 0.8,
                        n.rep = c(1000,10), seed = 10, trace = 2, cpus = 1)
> summary(e.n)
\end{Rcode}

\begin{Routput}
	Determination of the sample using a large sample (T=2000, C=2000)  

|++++++++++++++++++++++++++++++++++++++++++++++++++| 100% elapsed=06s  
- estimated effect (variance): 0.23748050.2612110.241430.24983750.2455650.238980.22754850.23496450.2462670.2383605 (1.2477591.2259471.2445871.2350031.239351.2433711.2564371.250751.2374461.24375)
- estimated sample size      : m=84, n=84

Simulation study with BuyseTest 

Simulation
- repetitions: 1000
- cpus       : 1

|++++++++++++++++++++++++++++++++++++++++++++++++++| 100% elapsed=16s  
\end{Routput}

\begin{Routput}
  Sample size calculation with Generalized pairwise comparison
  for a power of 0.8 and type 1 error rate of 0.05 

- estimated sample size (mean [min;max]): 84 [71;96] controls
                                          84 [71;96] treated

- net benefit statistic (null hypothesis Delta=0)
endpoint threshold n.T n.C mean.estimate sd.estimate mean.se rejection.rate
       Y     1e-12  84  84        0.2441      0.0864  0.0859          0.773

n.T          : number of observations in the treatment group
n.C          : number of observations in the control group
mean.estimate: average estimate over simulations
sd.estimate  : standard deviation of the estimate over simulations
mean.se      : average estimated standard error of the estimate over simulations
rejection    : frequency of the rejection of the null hypothesis over simulations
(standard error: H-projection of order 1| p-value: after transformation) 
\end{Routput}


\clearpage

\begin{align*}
	U - \Delta = \underbrace{\frac{1}{m} \sum_{i=1}^{m} h_E(i)}_{\text{Experimental group}}&  + \underbrace{\frac{1}{n} \sum_{j=1}^{n} h_C(j)}_{\text{Control group}} + \underbrace{ \frac{1}{m n}  \sum_{i=1}^{n}\sum_{j=1}^{m} h_{EC}(i,j)}_{\text{Second order term}} \notag \\
	\text{where for } i \in \{1,\ldots,m\}, h_E(i)&=\mathbb{E}[\mathds{1}_{Y^E_i > Y^C_j} - \mathds{1}_{Y^C_j > Y^E_i}\mid Y^E_i] - \Delta \notag \\
	j \in \{1,\ldots,n\}, h_C(j)&=\mathbb{E}[\mathds{1}_{Y^E_i > Y^C_j} - \mathds{1}_{Y^C_j > Y^E_i}\mid Y^C_j] - \Delta \notag  \\
 \widehat{\sigma}_{U} \underbrace{\approx}_{\text{First order}} \frac{1}{m^2} \sum_{i=1}^m h^2_E(i) &+ \frac{1}{n^2} \sum_{j=1}^n h^2_C(j)
\end{align*}

\begin{align*}
	p^{\mathcal{P}} = \frac{1}{1+P} \left\{1+\sum_{p=1}^P \mathds{1}_{\left| \Delta^{\mathcal{P}(p)}\right| \geq \left|\Delta \right|} \right\}
\end{align*}

\begin{align*}	\mathbb{P}\left[\textcolor{Orange}{Y^E_{i}}>\textcolor{\darkblue}{Y^C_{j}}|\textcolor{Orange}{\widetilde{Y}^E_{i}},\textcolor{Orange}{\Omega^E_i},\textcolor{\darkblue}{\widetilde{Y}^C_{j}},\textcolor{\darkblue}{\Omega^C_j}\right] = \left\{\begin{array}{c}
		0.75 \text{ for } i=i_1 \\
		1 \text{ for } i=i_2
	\end{array}\right.
\end{align*}

\begin{align*}
\left(
\textcolor{Orange}{\widetilde{Y}^E_{i_1}},
\textcolor{Orange}{\widetilde{Y}^E_{i_2}},
\textcolor{\darkblue}{\widetilde{Y}^C_{j}},
\textcolor{Orange}{\Omega^E_{i_1}},
\textcolor{Orange}{\Omega^E_{i_2}},
\textcolor{\darkblue}{\Omega^C_{j}}
\right) = (4.7,6.1,1.5,1,1,0)
\end{align*}

\(U_{ij}=\mathbb{P}\left[\textcolor{Orange}{Y_{i}}>\textcolor{\darkblue}{Y_{j}}+\tau|\textcolor{Orange}{\widetilde{y}_{i}},\textcolor{Orange}{\omega_i},\textcolor{\darkblue}{\widetilde{y}_{j}},\textcolor{\darkblue}{\omega_j}\right]-\mathbb{P}\left[\textcolor{\darkblue}{Y_{j}}>\textcolor{Orange}{Y_{i}}+\tau|\textcolor{Orange}{\widetilde{y}_{i}},\textcolor{Orange}{\omega_i},\textcolor{\darkblue}{\widetilde{y}_{j}},\textcolor{\darkblue}{\omega_j}\right]\)
\begin{table}[!ht]
	\centering
	\begin{tabular}{l@{}l@{}l|lll}
		(&$\Xcens$, & $\, \Ycens$) & $\xobs - \yobs \leq -\tau$ & $ |\xobs-\yobs| < \tau$ & $\xobs - \yobs \geq \tau$ \\ \hline 
		&&&&&\\ [-3mm]
		(&1,&1) & \(-1\) & \(0\) & \(1\) \\ [4mm]
		(&0,&1) & \(\frac{\Xsurv(\yobs+\tau)+\Xsurv(\yobs-\tau)}{\Xsurv(\xobs)}-1\) & $\frac{\Xsurv(\yobs + \tau)}{\Xsurv(\xobs)}$ & \(1\) \\ [4mm]
		(&1,&0) & \(-1\) &  $-\frac{\Ysurv(\xobs + \tau)}{\Ysurv(\yobs)}$ & \(1-\frac{\Ysurv(\xobs+\tau)+\Ysurv(\xobs-\tau)}{\Ysurv(\yobs)}\) \\ [4mm]
		(&0,&0) & \(\ldots\)
		& \(\frac{\int_{\xobs}^{\infty} \Ysurv(t+\tau)d\Xsurv(t)-\int_{\yobs}^{\infty} \Xsurv(t+\tau)d\Ysurv(t)}{\Xsurv(\xobs)\Ysurv(\yobs)}\)
		& \(\ldots\)
		\\ [-4mm]
		&&&&&\\ \hline
	\end{tabular}
\end{table}
%\(-\frac{\int_{t>\yobs}^{\infty} \Xsurv(t+\tau) d\Ysurv(t)}{\Xsurv(\xobs)\Ysurv(\yobs)}\)
% \(-\frac{\int_{t>\yobs}^{\infty} \Xsurv(t+\tau) d\Ysurv(t)}{\Xsurv(\xobs)\Ysurv(\yobs)}\)
% \( 1 - \frac{\Ysurv(\xobs-\tau)}{\Ysurv(\yobs)} - \frac{\int_{t>\xobs-\tau}^{\infty} \Xsurv(t+\tau) d\Ysurv(t)}{\Xsurv(\xobs)\Ysurv(\yobs)}\)

\begin{align*}
	\mathbb{P}\left[\textcolor{Orange}{Y^E}>\textcolor{\darkblue}{Y^C}\right]-\mathbb{P}\left[\textcolor{\darkblue}{Y^C}>\textcolor{Orange}{Y^E}\right]
\end{align*}

\begin{align*}
	\mathbb{P}\left[\textcolor{Orange}{Y^E}>\textcolor{\darkblue}{Y^C}\right]+0.5\mathbb{P}\left[\textcolor{Orange}{Y^E}=\textcolor{\darkblue}{Y^C}\right]
\end{align*}

\begin{align*}
	\frac{\mathbb{P}\left[\textcolor{Orange}{Y^E}>\textcolor{\darkblue}{Y^C}\right]+0.5\mathbb{P}\left[\textcolor{Orange}{Y^E}=\textcolor{\darkblue}{Y^C}\right]}{
	\mathbb{P}\left[\textcolor{\darkblue}{Y^C}>\textcolor{Orange}{Y^E}\right]+0.5\mathbb{P}\left[\textcolor{\darkblue}{Y^C}=\textcolor{Orange}{Y^E}\right]}
\end{align*}

\begin{align*}
	\frac{\mathbb{P}\left[\textcolor{Orange}{Y^E}>\textcolor{\darkblue}{Y^C}\right]}{\mathbb{P}\left[\textcolor{\darkblue}{Y^C}>\textcolor{Orange}{Y^E}\right]}
\end{align*}


\begin{align*}
	\delta_1 = &\mathbb{P}\left[\textcolor{Orange}{Y_1^E}>\textcolor{\darkblue}{Y_1^C}+\tau_1\right]-\mathbb{P}\left[\textcolor{\darkblue}{Y_1^C}>\textcolor{Orange}{Y_1^E}+\tau_1\right] \\
	\delta_2 = &\mathbb{P}\left[\textcolor{Orange}{Y_2^E}>\textcolor{\darkblue}{Y_2^C}\big| \, |\textcolor{Orange}{Y_1^E}-\textcolor{\darkblue}{Y_1^C}| \leq \tau_1\right]-\mathbb{P}\left[\textcolor{\darkblue}{Y_2^C}>\textcolor{Orange}{Y_2^E}\big| \, |\textcolor{Orange}{Y_1^E}-\textcolor{\darkblue}{Y_1^C}| \leq \tau_1\right] 
\end{align*}



\begin{align*}
	\Delta_1 &= \delta_1 \\
		\Delta_2 &= \delta_1 + \delta_2
\end{align*}

\begin{minipage}{3.5cm}
\begin{tcolorbox}[ams align*, colback=white, colframe=\darkgreen]
	\Delta_1 &= \delta_1 \\
	\Delta_2 &= \delta_1 + \delta_2
\end{tcolorbox}
\end{minipage}

\bigskip

\begin{minipage}{14cm}
	\begin{tcolorbox}[ams align*, colback=white, colframe=colour_pink]
	\delta_1 = &\mathbb{P}\left[\textcolor{Orange}{Y_1^E}>\textcolor{\darkblue}{Y_1^C}+\tau_1\right]-\mathbb{P}\left[\textcolor{\darkblue}{Y_1^C}>\textcolor{Orange}{Y_1^E}+\tau_1\right] \\
	\delta_2 = &\mathbb{P}\left[\textcolor{Orange}{Y_2^E}>\textcolor{\darkblue}{Y_2^C}\big| \, |\textcolor{Orange}{Y_1^E}-\textcolor{\darkblue}{Y_1^C}| \leq \tau_1\right]-\mathbb{P}\left[\textcolor{\darkblue}{Y_2^C}>\textcolor{Orange}{Y_2^E}\big| \, |\textcolor{Orange}{Y_1^E}-\textcolor{\darkblue}{Y_1^C}| \leq \tau_1\right] 
	\end{tcolorbox}
\end{minipage}

\end{document}